%
%      author : Louis Cordier <lcordier@gmail.com>
%     license : Creative Commons, Attribution (by)
%             :
% description : This is the root LaTeX file for an general (journal) article.
%             :
%      layout : My proposed default.
%             :
%             : title
%             : abstract
%             : introduction
%             : 
%             : conclusions
%             : bibliography
%             :
%    modified : $Id$
%

\documentclass[10pt,conference,a4article]{IEEEtran}

\newif\ifpdf 
\ifx\pdfoutput\undefined 
  \pdffalse      % we are not running PDFLaTeX
\else 
 \pdfoutput=1    % we are running PDFLaTeX 
 \pdftrue
\fi

\ifpdf
\usepackage[pdftex,
            backref=true,
	    pagebackref=true,
	    pdfpagemode=None,
	    colorlinks=true,
            plainpages=false,
            pdftitle=Pretty\ Frecking\ Strong\ Encryption,
            pdfauthor=Louis\ Cordier,
            pdfsubject=Cryptography,
            pdfkeywords=cryptographic\ key-distribution,
            pdfstartview=FitH,
            linkcolor=blue,
            citecolor=blue,
            urlcolor=blue]{hyperref}
\else
\usepackage[dvips,
            backref=true,
	    pagebackref=true,
	    pdfpagemode=None,
	    colorlinks=true,
            plainpages=false]{hyperref}
\fi




\begin{document}

\title{Pretty Frecking Strong Encryption}
\author{Louis Cordier {\tt<}lcordier@gmail.com{\tt>}\\6 September 2009}
\date{6 September 2009}
\maketitle

\begin{abstract}
In this paper I'll show that we can address the key distribution 
problem of one-time pads in a viable and practical manner. Instead
of distributing a key, we will (publically) distribute a recipe to
make a key. The two parties communicating will have shared prior
knowledge of a secret ingredient used in the construction of the
key. The attacker, armed with only knowledge of the cipher-text and
the recipe, will have to guess the secret ingredient. I will then show
that the secret-ingredient-space is with all likelihood in the order 
of aleph-null.
\end{abstract}

\section{Introduction}
In 1917 Gilbert Vernam invented the one-time pad \cite{OTP}, later
Claud Shannon showed it had a property of {\em perfect secrecy}.
For this to work, it is vital that the keystream values (a) be truly
random and (b) never be reused \cite{OTPblog}. Additionally, the keystream
must be as long as the data to be encrypted and has to be
shared {\em securely} with the recipient. Sharing the key stream 
securely is known as the Key Distribution problem \cite{KeyDistribution}.

\section{Assumptions}
We will assume all clear text messages will always be compressed
at the maximum compression ratio to increase message entropy.

\section{Key Recipes}
A key recipe will be a string that references public resources (ingredients)
and specify the size of the key to be created. Recipies can be encoded
in any language, obfuscated in any format. However, we will assume the 
attacker will always know the format and language in which recipies are
encoded, and always have access to all recipes. 

\subsection{URL Recipe}
For my first recipe encoding scheme, I decided to stick to plain URLs. In 
general we will use the fragment of an URI. In RFC3986 section 3.5 \cite{RFC3986}
--- a fragment identifier component is indicated by the presence of a number
sign (``\#'') character and terminated by the end of the URI. Thus we can embed
URLs (without fragment sections) into the fragment section of an URI.
These URLs can be delimited with a number sign (``\#''). For example, an URL 
encoded recipe could look like:
\begin{itemize}
\item{\url{http://www.google.com/#http://example.com/ingredient1#http://example.com/ingredient2}}
\end{itemize}
This recipe can then be obfuscated using URL-shorteners, for example tinyurl
and bit.ly. Thus the example recipe would now look like:
\begin{itemize}
\item{\url{http://tinyurl.com/nhzyue}}
\item{\url{http://bit.ly/NKsoX}}
\end{itemize}
If we always use the same URL-shortener (even privately controlled ones), we simply
have to send the last 5 characters. We could SMS it, tweet it, hide it \cite{DeCSS},
\cite{Steganography}, \cite{SpreadSpectrum}, morse code it or even publish it as a 
coupon code in a newspaper ad.

\subsection{Email Recipe}
Email recipes can be written in natural languages wich will make automated techniques
to parse them difficult. For example:
\begin{verbatim}
From: Bob <bob@example.com>
To: Alice <alice@example.com>

Here is a nice article you might like.
http://www.example.com/recipe.html
Use the first two images on the page as
your ingredients, and that hot image
of you at summer camp as the secret
ingredient ;).

Regards, Bob.
\end{verbatim}

\section{Key Ingredients}
Ingredients can be thought of as data, sampled in an endless loop. The loop is not 
really infinite, it just loops until we gathered enought values to construct the
required length key. Ingredients can also be code or algorithms that generate 
values, PRNGs \cite{PRNG} for example.\\[0.25ex]
\begin{quote}
Progression of a Lisp programmer --- the newbie realizes that the difference between
code and data is trivial. The expert realizes that all code is data. And the true
master realizes that all data is code.
\end{quote}

\subsection{Conditional Access Ingredients}
Since ingredients will most likely be stored on public webservers, we could implement
features like access control. Think of an AppEngine application that returns a file of
requested size, $N$. The file is dynamically created as the first $N$ outcomes of an PRNG,
optionally modulated by some algorithm/state-machine, seeded with the requesting 
machine's IP address. If Alice is constructing a key and requesting her ingredients 
over SSL the attacker will have to request his own set of ingredients. If the public 
webserver shows stats of how many times resources have been accessed, Alice can tell 
if an attacker is trying to construct her key. An example of this would be to see how
many people accessed our recipe.
\begin{itemize}
\item{\url{http://bit.ly/info/NKsoX}}
\end{itemize}

\subsection{The Secret Ingredient}
Ingredients can be both data and code. The secret ingredient, the one only Alice and
Bob knows, should ideally be some algorithm. This algorithm takes a keystream $K_i$,
constructed from the ingredients listed in the recipe and mutates, transforms or 
transcode it into our final keystream $K'_i$. $K'_i$ is the keystream we use to encode
and decode our message. The attacker will have to determine a secret algorithm of
arbitrary complexity. To make things even more difficult, within each message we can
include a new secret ingredient (algorithm) to be used in the response message.
These algorithms can be general functions with sets of parameters that control their
behaviour. The controling parameters can then be selected at random within bounds. 
Here is a few example secret ingredients:
\\
\subsubsection{Simple Function}
Our secret ingredient is the function $F$ such that $K'_i = F(K_i)$. Where:
$K'_i = (K_i + 3)\;MOD\;255$ when $i$ even and $K'_i = (K_i + 5)\;MOD\;255$ when 
$i$ is odd. The function $F$ is secret, but it is also in a general form with the
parameters (3,5) that we could have chosen at random. So instead of sending a new
secret function $F$ in each message, we could simply only send new contol parameters.
\\
\subsubsection{PRNG}
In this case our secret ingredient is a function that interacts with a PRNG, lets
say Mersenne Twister. We seed it with an secret parameter $s$ and we ignore the 
first $n$ outcomes. Then $F$ is the function such that $K'_i =
K_i\oplus MT(s)_{i+n}$ where $MT$ is the Mersenne Twister with outcomes bound to
the bit-width of $K_i$.

\section{Key Construction}
OTPs require that the key be of the same length as the original message. For a 1TB
message we will need a 1TB key. Thus a length-limited recipe must be turned into an
arbitrary length key. Think of this as an inverse-one-way-hash function, given a
hash (recipe) it will return the minimum-sized data that will produce that hash
(desired key). Constructing a key is the process of mixing ingredients in their 
various quantities, and then altering the mixture in some way (baking it) to
produce the final result.

\subsection{Mixing Ingredients}
The easiest way to mix ingredients would be to cyclicly multiplex them. For fixed
length data, put it in a circular buffer of the same size as the data. Then to
generate the intermediate keystream $K_i$ simply select, in a cyclic ordered
fasion, values from each ingredient buffer. Thus we have a length-unlimited keystream.
\\[2ex]
If the ingredient is an algorithm, simply pass it an unlimited time-series as
input. For example the sequence $1,2,3,\dots$
\\[2ex]
If our secret ingredient is data instead of an algorithm, say an image taken on
a digital camera when Bob and Alice are together, we could multiplex it with our
other ingredients. We then wouldn't need to alter the keystream further, $K'_i = K_i$.

\subsection{Baking The Mixture}
Since the attacker has our recipe, he can with all likelihood always construct the 
intermediate keystream $K_i$. It is therefore vital that we alter $K_i$ in some secret
way to produce our key $K'_i$. We require that it MUST be computationally unfeasible
for the attacker to find our secret. To that end we choose an arbitrary algorithm
to mutate $K_i$ into $K'_i$. I would venture to guess that there are at least 
as many possible algorithms as there are integers and way more than Prime numbers.
Thus my secret ingredient search space is at least of the order $Aleph_0$.
\\[2ex]
We can write it as equations:
\begin{eqnarray}
K_i &=& Mux(I_1,I_2,\dots,I_n,i) \\
K'_i &=& F(K_i,i,\dots)
\end{eqnarray}
Which states the intermediate keystream $K_i$ is the multiplexed ingredients 
$I_1,I_2,\dots,I_n$, and the final keystream $K'_i$ is the result of a mutation function
$F$, our secret ingredient, acting on the intermediate keystream. 

\section{Conclusion}
I have shown that the key distribution problem could trivially be addressed with the
(public) distribution of key generation recipes and a shared secret. The shared 
secret could be data or an algorithm and its size extremely small compared to the 
message to be encryptred. I also showed a protocol where new shared secrets get 
distributed with each message, to be used in the response of that message. Thus to 
establish and maintain {\em perfect secrecy} you only need to distribute one 
{\em small} secret. That secret could be obfuscated \cite{DeCSS} and hidden
\cite{Hardware} in physical ``unbreakable'' objects, couriered around the world.
\\[2ex]
This leaves us with a moral dilemma. Should this genie be left out of the bottle? 
I could patent it in such legal speak that no one could implement it, and then sell
various hardware solutions to governments to be used by their agents in the field.
If I don't release this, someone else is bound to come up with a similar solution. 
Should the public, terrorists and pedophiles have perfect secrecy?

\bibliographystyle{IEEEtran}
\bibliography{lcordier}

\end{document}
